\documentclass{beamer}

%
%retrieved from https://www.overleaf.com/latex/templates/beamer-presentation/zxrfltwmbcrt#.Vyig856rSkA 03.05.2016
% Choose how your presentation looks.
%
% For more themes, color themes and font themes, see:
% http://deic.uab.es/~iblanes/beamer_gallery/index_by_theme.html
%
\mode<presentation>
{
  \usetheme{default}      % or try Darmstadt, Madrid, Warsaw, ...
  \usecolortheme{default} % or try albatross, beaver, crane, ...
  \usefonttheme{default}  % or try serif, structurebold, ...
  \setbeamertemplate{navigation symbols}{}
  \setbeamertemplate{caption}[numbered]
} 

\usepackage[english]{babel}
\usepackage[utf8x]{inputenc}

\title[Geostatistics]{Introduction to Data Management}
\author{Sergio Vargas}
\institute{Dept. Earth \& Environmental Sciences, LMU M\"unchen}
\date{2016}

\begin{document}

\begin{frame}
  \titlepage
\end{frame}

% Uncomment these lines for an automatically generated outline.
%\begin{frame}{Outline}
%  \tableofcontents
%\end{frame}

\section{Data management}

\begin{frame}{Things you need}
\begin{enumerate}
  \setbeamercovered{transparent=0}
  \item Spreadsheet:
    \begin{itemize}
      \item \textbf{LibreOffice}: Why to use LibreOffice? Ideas.
      \pause
      \item \textbf{Always} save your files as comma (or tab) separated value (csv) files. Why?
    \end{itemize}
\end{enumerate}
\end{frame}


\begin{frame}{Metadata}
  What is it?
  \pause
  Data about data.
  \begin{enumerate}
    \item Name and contact information of the person in charge of the study.
    \item Where were the data collected.
    \item Study name.
    \item Funding body.
    \item Information about the data file:
  \end{enumerate}
\end{frame}


\begin{frame}{Information about the data file:}
  \begin{itemize}
    \item Type of experimental unit.
    \item Methods use to collect the data.
    \item Units of measurement or observation of each of the variables.
    \item Abbreviations used in the file.
    \item Description of what data are in the columns and the rows.
    \item Character used to separate the columns.
    \item Character used to separate the rows.
  \end{itemize}
\end{frame}


\begin{frame}{Data storage in the 21th century}
  Traditional way:
  
    \begin{itemize}
      \item Make a copy of the data and store it somewhere on you computer.
    \end{itemize}
 
  \pause
  
  Problems?
  
  {\tiny Data versioning, for example.}

\end{frame}

\begin{frame}{Data versioning: the old way}

  Suppose I have file A and want to modify something.
  
  \textbf{Procedure 1}

  \begin{itemize}
  
    \item Open file.
    \item Modify file.
    \item Save file.
    \item Close file.
  
  \end{itemize}

  Problems?
  
\end{frame}
  
\begin{frame}{Data versioning: the old way}

  \textbf{Procedure 2}
  
  \begin{itemize}
  
    \item Duplicate file.
    \item Rename new/old file.
    \item Open file.
    \item Modify file.
    \item Save file.
    \item Close file.
  
  \end{itemize}

  Problem?

\end{frame}

\begin{frame}{Data versioning: a new way}

  {\Large \textbf {Git}}
  
  Git is a version control system that is:
  
  \begin{itemize}
  
    \item distributed
    \item emphasizes on speed and data integrity
    \item supports non-linear workflows
    \item nowadays, is ``kinda'' cloud-based \newline {\tiny (storage and accesibility is accounted for while you keep a copy of the file on your system)}
  
  \end{itemize}
  
  Example:

    \href{https://github.com/sevragorgia/Statistics}{Github}
    
    BitBucket
    
\end{frame}

\begin{frame}{Benefits of Git}

  Git is like a super advanced Save-As system. But...

  \pause

\begin{itemize}

  \item Changes done to a file can be tracked back and undo.
  
  \pause
  
  \item Allows collaboration with other researchers.
  
  \pause
  
  \item Allows the creation of sandboxes (called branches) to modify copies of files.
  
  \pause
  
  \item All this, while keeping only one copy of the file (and records of the changes done to it).

\end{itemize}

\end{frame}

\begin{frame}{Demo: create a repository for the data used in this practical}

  \begin{enumerate}
  
    \item Create a new folder to store the contents of the project.
    \pause
    \item Before you add any data or move files or anything, initialize git.
    \pause
    \item Initialize git (git init folder\_name).
    \pause
    \item Add data and start tracking changes.
    \pause
    \item Once you are happy with the results, publish it on e.g. Github.
    \pause
    \item Branch the published version of the data to keep working on it while other could potentially work on the published version.
    \pause
    \item Merge the changes to the already published version of the data; other users of the data will keep their versions unless they try to compare their version against the newly available one.

 \end{enumerate}

\end{frame}


\end{document}
