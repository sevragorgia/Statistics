\documentclass{beamer}

%
%retrieved from https://www.overleaf.com/latex/templates/beamer-presentation/zxrfltwmbcrt#.Vyig856rSkA 03.05.2016
% Choose how your presentation looks.
%
% For more themes, color themes and font themes, see:
% http://deic.uab.es/~iblanes/beamer_gallery/index_by_theme.html
%
\mode<presentation>
{
  \usetheme{default}      % or try Darmstadt, Madrid, Warsaw, ...
  \usecolortheme{default} % or try albatross, beaver, crane, ...
  \usefonttheme{default}  % or try serif, structurebold, ...
  \setbeamertemplate{navigation symbols}{}
  \setbeamertemplate{caption}[numbered]
} 

\usepackage[english]{babel}
\usepackage[utf8x]{inputenc}

\title[Geostatistics]{Introduction to Data Management}
\author{Sergio Vargas}
\institute{Dept. Earth \& Environmental Sciences, LMU M\"unchen}
\date{2016}

\begin{document}

\begin{frame}
  \titlepage
\end{frame}

% Uncomment these lines for an automatically generated outline.
%\begin{frame}{Outline}
%  \tableofcontents
%\end{frame}

\section{Data management}

\begin{frame}{Things you need}
\begin{enumerate}
  \setbeamercovered{transparent=0}
  \item Spreadsheet:
    \begin{itemize}
      \item \textbf{LibreOffice}: Why to use LibreOffice? Ideas.
      \pause
      \item \textbf{Always} save your files as comma (or tab) separated value (csv) files. Why?
    \end{itemize}
\end{enumerate}
\end{frame}


\begin{frame}{Metadata}
  What is it?
  \pause
  Data about data.
  \begin{enumerate}
    \item Name and contact information of the person in charge of the study.
    \item Where were the data collected.
    \item Study name.
    \item Funding body.
    \item Information about the data file:
  \end{enumerate}
\end{frame}


\begin{frame}{Information about the data file:}
  \begin{itemize}
    \item Type of experimental unit.
    \item Methods use to collect the data.
    \item Units of measurement or observation of each of the variables.
    \item Abbreviations used in the file.
    \item Description of what data are in the columns and the rows.
    \item Character used to separate the columns.
    \item Character used to separate the rows.
  \end{itemize}
\end{frame}


\begin{frame}{Data storage in the 21th century}
  Traditional way:
  
    \begin{itemize}
      \item Make a copy of the data and store it somewhere on you computer.
    \end{itemize}
 
  \pause
  
  Problems?
  
  \tiny{Data versioning, for example.}

\end{frame}

\begin{frame}{Data versioning: the old way}

  Suppose I have file A and want to modify something.
  
  \textbf{Procedure 1}

  \begin{itemize}
  
    \item Open file.
    \item Modify file.
    \item Save file.
    \item Close file.
  
  \end{itemize}

  Problems?
  
\end{frame}
  
\begin{frame}{Data versioning: the old way}

  \textbf{Procedure 2}
  
  \begin{itemize}
  
    \item Duplicate file.
    \item Rename new/old file.
    \item Open file.
    \item Modify file.
    \item Save file.
    \item Close file.
  
  \end{itemize}

  Problem?

\end{frame}

\begin{frame}{Data versioning: a new way}

  \Large{\textbf {Git}}
  
  Git is version control system that is:
  
  \begin{itemize}
  
    \item distributed
    \item emphasizes on speed and data integrity
    \item supports non-linear workflows
    \item nowadays, is kinda cloud-based
  
  \end{itemize}
  
  Example:

\end{frame}

\end{document}
