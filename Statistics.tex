\documentclass{beamer}

%
%retrieved from https://www.overleaf.com/latex/templates/beamer-presentation/zxrfltwmbcrt#.Vyig856rSkA 03.05.2016
% Choose how your presentation looks.
%
% For more themes, color themes and font themes, see:
% http://deic.uab.es/~iblanes/beamer_gallery/index_by_theme.html
%
\mode<presentation>
{
  \usetheme{default}      % or try Darmstadt, Madrid, Warsaw, ...
  \usecolortheme{default} % or try albatross, beaver, crane, ...
  \usefonttheme{default}  % or try serif, structurebold, ...
  \setbeamertemplate{navigation symbols}{}
  \setbeamertemplate{caption}[numbered]
} 

\usepackage[english]{babel}
\usepackage[utf8x]{inputenc}

\title[Geostatistics]{Introduction to Statistics for Geo- and Paleobiologist}
\author{Sergio Vargas}
\institute{Dept. Earth \& Environmental Sciences}
\date{2016}

\begin{document}

\begin{frame}
  \titlepage
\end{frame}

% Uncomment these lines for an automatically generated outline.
%\begin{frame}{Outline}
%  \tableofcontents
%\end{frame}

\section{Announcement}

\begin{frame}{Possible Research Project/M.Sc. Projects}

\begin{itemize}
  \setbeamercovered{transparent=25}
  \item Characterization of \emph{C. auris} bacterial communities: 16S amplicon sequencing, artificially long reads (MSc) to close the cyanobacterial genome.
  \pause
  \item Characterization of \emph{Spongilla lacustris} bacterial communities found in gemmulae: 16S amplicon sequencing of single gemmulae. Depending on the richness the project could be developed in different directions.
  \pause
  \item Dynamics of sclerite formation in \emph{Pinnigorgia flava}: tracking sclerite formation in time and space under different ocean simulations.
  \pause
  \item Characterization of \emph{Sarcothelia edmonsoni}: molecular identification using mtMutS sequences, characterization of the symbionts using molecular methods, characterization of the growth form of the colony: bifurcation patterns of the ``stolons''. Could be further developed as a MSc project.
\end{itemize}
\end{frame}
\end{document}
